\documentclass[12pt]{article}

\usepackage{amsmath}    % For mathematical symbols and equations
\usepackage{amssymb}    % For additional math symbols
\usepackage{geometry}   % To control page margins
\usepackage{hyperref}   % To include hyperlinks
\usepackage{fancyhdr}   % To customize the header and footer
\usepackage{enumitem}   % For better lists
\usepackage{setspace}   % For line spacing
\usepackage{parskip}    % For better paragraph spacing

\geometry{a4paper, margin=1in}

\pagestyle{fancy}
\fancyhf{}
\fancyhead[L]{Computer Architecture} 
\fancyfoot[C]{\thepage}

\title{Introduction to Computer Architecture}
\author{0xSaad / Saad Almalki}
\date{\today}

\begin{document}

\maketitle

\section{Logic Circuits}
Logic circuits are the building blocks of digital systems. They consist of interconnected logic gates that perform logical operations on binary inputs to produce a binary output.

There are two types:
\begin{enumerate}[label=\arabic*.]
    \item \textbf{Combinational:} Adder, Subtractor, Decoder, Encoder, Comparator, Multiplexer, Demultiplexer.
    \item \textbf{Sequential:} Register, Flip-Flop, Latches, Counters.
\end{enumerate}

\section{Microoperation}
Operations will be performed on data stored in registers.  
\textbf{Example:} \texttt{SHIFT} -- \texttt{LOAD} -- \texttt{CLEAR} -- \texttt{INCREMENT}

\section{Register Transfer Language}
Register Transfer Language (RTL) is a symbolic representation of the operations performed on registers in a computer system. It describes how data is transferred between registers and how operations are executed.

\vspace{0.5em}
Example: \quad \texttt{R2 $\rightarrow$ R1}

\vspace{0.5em}
We can also add a control function; the register cannot transfer without satisfying this condition:

\vspace{0.5em}
\texttt{P: R2 $\rightarrow$ R1}

\section{Register Names}
The register names can have specific meanings:
\begin{itemize}
    \item \textbf{DR:} Data Register
    \item \textbf{IR:} Instruction Register
    \item \textbf{TR:} Temporary Register
    \item \textbf{AC:} Accumulator
    \item \textbf{PC:} Program Counter
    \item \textbf{MAR:} Memory Address Register
\end{itemize}

\begin{flushright}
Created by: Saad Almalki \\
\end{flushright}

\end{document}
